\chapter{Requirements}
\label{ch:requirements}

The RDM team will develop a phased plan that will satisfy the requirements listed below
and will meet the milestones discussed in Section~\ref{sec:Milestones}.


\begin{enumerate}
\item Move data from DLLS to long term storage in a timely fashion.
  \begin{enumerate}
  \item In normal operations this should be complete within N minutes of the end of data taking for that subrun. \fixme{N=5? 10? 15?}.
  \item 95\% of the time, Pass1 for a subrun must be complete within 6 hours of the end of data taking for that subrun.
  \item The TDAQ will produce some files that need to travel in pairs, for example Trk+Cal on-spill file and CRV on-spill file.
    Prioritize data handling to move these together.
  \item When dCache is not available, data will pile up in the DLLS.  When service is restored prioritize data handling of the backlog
    so that the highest priority files are copied first.
  \end{enumerate}
\item Most data sets should be written to a file family that makes two copies on tape in physically separate locations.
  There may be some data sets for which a single copy is adequate.
  Work with the collaboration to decide which data set falls into which category.
\item Update the file catalog to provide meta-data and locations for all new files.
\item Copy/mirror information in the various online databases to the appropriate offline databases.
  \begin{enumerate}
  \item Prioritize any information that is needed for Pass1; an example is that Pass1 may want the conditions information
    used by the trigger filter processes.
  \end{enumerate}
\item Copy other files from online disks to long term storage; examples include log files generated by online
  operations and files produced by online DQM.
\item Manage the free space in the DLLS.  This is adapted from~\cite{OnlineMonitoring}.
  \begin{enumerate}
   \item Ensure that there is adequate free space to write N more hours of data (maybe 6 hours?).
   \item Keep track of which files are deletable because they are already on tape or otherwise have redundant copies in the offline world.
   \item Periodically delete files to ensure a)
   \item Provide a mechanism to keep some files on disk in the online system for an extended period so that
     people can do studies on them using online resources. One way is to provide
     a method to pin files in the DLLS.  An alternate solution is to carve out a different piece of disk
     space for this use.
     What ever solution is chosen, it must not disrupt data taking.
   \item Tell OTS DAQ to stop data taking if the disk is full.
  \end{enumerate}
\item Manage free space in persistent dCache and other offline storage resources that we use.
\item Provide a way to control and monitor all parts of the RDM; the monitoring system should raise warnings or alarms when appropriate.
\item If Mu2e needs to split/join or otherwise reshape the data from the DLLS before it is written to tape, do this work.
  This task will be performed using offline resources because there are no resources in the online world to do it.
\item If needed, inform the follow-on workflows when data is ready for them.
\end{enumerate}

Some comments about the requirements.

Why 6b) and 6c)?
Why not just delete files from the DLLS as soon as the are on tape?
The use case for this is commissioning and debugging.
There will be situations in which we have data in the DLLS and have stopped taking data
until an issue is resolved.
In this case there are cycles available on the DAQ servers to study the issue.
This is a more powerful resource than our interactive GPVM nodes and
it will have faster turn around than running grid jobs.
It is also not subject to scheduled maintenance at inconvenient times.
It seems prudent to keep the maximum amount of data available in the DLLS so that
it is available for understanding emergencies.
