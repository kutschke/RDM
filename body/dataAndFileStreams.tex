\chapter{Data Streams and File Streams}
\label{ch:DataStreamsAndFileStreams}

This chapter defines the ideas of ``data streams'' and ``file streams''.
It enumerates the data streams that are anticipated at this time
and it presents a strawman for how these may be grouped into files.

\fixme{Should I use the word ``data set'' instead of ``file stream''?
  I am using a distinct word for now since I want to reserve the word
  of data set for something that is defined using SAM.  Are these
  two ideas identical?
}

A ``data stream'' is a series of events selected by some algorithm;
one attribute of events in a data stream is they all require the same downstream processing.
For example the on-spill and off-spill events are distinct data streams.
The on-spill events will have many trigger lines; one might consider each trigger
line it's own data stream or one might define groups of trigger lines as a data stream.
Another example: the off-spill events will include some events that triggered based on
the tracker or calorimeter and they will also include SiPM noise events.
The triggered events and the SiPM noise events are distinct data streams.

A ``file stream'' is a series of files that contain events from one or more data streams.
For example we might decide to bundle all of the on-spill data streams into a single
file stream or we might decide to put some of the on-spill data streams into one file
stream and another set of on-spill data streams into a different file stream.
For example all of the files containing off-spill events might be one file stream;
or maybe we will decide to split the off-spill events in to
the two off-spill data streams into two file streams.

The key point is that data streams are a physics driven idea,
while file streams are choice about how to group data streams for convenience of data handling and downstream processing.

\fixme{We need to address if a given event may be in more than one data stream.
       And we need to address if a given data stream may be in more than one file stream.
}

The design of the grouping of data streams in to file streams
is in progress.
In an ideal world the TDAQ would chose to organize data into file streams that best match the needs
of downstream processing.
This pushes the decision towards more file streams.
However there are overheads associated with each file stream and that may force Mu2e to fewer file streams
than optimal for downstream processing.
If this happens, the DPC organization will need to decide where in the workflow to perform
the reorganization of the data.
