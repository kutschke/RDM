\chapter{Questions}
\label{ch:questions}

Below is a list of questions that must be addressed during the design of the RDM:

\fixme{These are not ordered in any way.  How should we order them?}

\begin{enumerate}
  \item Be sure that all stakeholders are represented when making decisions.
  \item Eric Flumerfelt says that he envisages the gateway machine as a minimal machine.  Will this meet our needs?
    If not then we need spec an appropriate machine and work with the TDAQ L2 and the TDAQ (pre)Ops to ensure an appropriate
    machine. It's very likely that we can live with a minimal machine until the start of commissioning with beam.
  \item Is there plan for a hot spare for the gateway machine? If not, what about a cold spare?
    This will be the responsibility of TDAQ (pre) Ops.  We need an understanding with them.
  \item I understand that the machines in the TDAQ farm, including the gateway machine, will be admin by SCD personnel.  Make sure that
    RDM interests are presented in agreements with SCD.
  \item Do we need dedicated computing resources in the computer center? Is it good enough to request additional VMs configured similar to
    the mu2egpvm machines?  If not, what are our options?  In all cases understand who admins the machines, hot spares or cold spares.
    Remember that the reason that SCD likes to support VMs is that it's very easy to stand up a replacement if a host dies.
  \item Be sure to plan one or more design reviews and one or more Operational Readiness reviews.
  \item The TDAQ team plans that each of the trigger filter processes will have a unique process name.
    Also each of the {\code artdaq::DataLogger} processes will have a unique name.
    This may break the current \art implementation of secondary input files; see Section~\ref{appsec:SecondaryInputs}.
    We should develop a proof of principle demonstration as soon as the TDAQ team can provide examples of files written by
    the  {\code artdaq::DataLogger}.
  \item The ExtMon summary data will be small, on the scale of 200~KiB per MI cycle.  How will this be aggregated for storage on tape?
    The answer may depend on how it is used in downstream processing.  Is SFA good enough?
  \item Decide on a split level for data products in \art root files.  We now use 99999; CMS uses 0 (or 1, whichever is the smallest allowed).
  \item Do we need to extend our file families.  For example does it make sense to store on-spill triggered events on the same tapes as SiPM pedestal data?
  \item Work with the collaboration to understand the downstream processing plans for all data streams.  Use this to inform what we do with
    files once they are in dCache,

  \item How will Pass1 and other follow-on workflows know that new data has arrived?
    Will it be driven entirely by updates to the file catalog or do we need other
    hooks to those workflows?
  \item Does it make sense to consider POMS to drive the workflow of RDM?
  \item Does RDM have any responsibilities to move information from EPICS to offline?
  \item Mu2e needs to manage the free space in /pnfs/mu2e/persistent. Many parts of DPC will touch this space.
    Whose job is it to ensure what there is adequate free space? Maybe RDM should have a separate dCache
    partition to isolate it from issues caused by other users?
  \item What does it mean for a file to be ``on tape'' and therefore deletable?
    For example, if it's in both persistent and tape backed dCache, but not yet on tape, is that good enough?
  \item Work with TDAQ to define a policy for locating files within the DLLS.
  \item Revisit the question of keeping N hours of free space in the DLLS vs aggressively deleting files to keep
    maximum free space.  If we decide to change then we need to update the requirements.
  \item Update data size estimates; using the design DLLS size, how many hours of buffer do we have?  Is it good enough?
    If not, do we ask TDAQ to buy more space or do we live with a shorter maximum buffering time?
    The size of the buffer will not be an issue until we are commissioning with beam so this does not need to
    be addressed soon and it need not be on-project.
  \item Work with the TDAQ team to define a protocol for TDAQ to inform us when new files are available.
    My instinct is that we don't to walk the file system discover files that need to be moved to tape.
    I prefer that we receive a message and manage a queue; this implies persistency for the queue state
    and error recovery tools.
    Should this be raised to a requirement?
  \item Think about recovering from from a long network or dCache downtime:
    we will want to identify pairs of a Trk+Cal file and its matching CRV file and move these pairs as a unit;
    we do not want catch up processing delayed while one file is on dCache but it's partner is not.
    Do have a operations based preference for LIFO vs FIFO? What about all on-spill before any off-spill?
  \item Corner cases to manage:
    \begin{enumerate}
      \item Files or subruns that contain zero events.
      \item If data taking is paused, does RMD need to know that?  If so, how will it be notified?
      \item When data taking has stopped for more than a few minutes (say a 15 minute maintenance access ),
        we probably want to flush the system to make sure that all data is processed to the end of Pass1
        in a timely fashion.
        This is only relevant if there are synchronization points in the workflows in which processing
        subrun N waits subrun N+1 before it takes the next step.
      \item A variant on the previous item.  Suppose we are making tar files to archive read-rarely small
        files, such as the ExtMon summary data.
        If there is a pause in data taking, when do we decide to make a small tarball and when
        do we wait for more files?  And the same question if we choose an aggregation method that is not
        tar files.
      \item If a shifter marks a run as bad, do we still want to run Pass1?
    \end{enumerate}
  \item   When do we move SAM to RUCIO? As soon as we can? During the long shutdown or later?  Never?
    How will an early-ish transition interact with requirements for ongoing simulation, VST and test stand work?
    Can both SAM and RUCIO coexist?
  \item Work with TDAQ to understand the constraints that drive the choices of how to
    map data streams to file streams.
    \begin{enumerate}
      \item A separate file stream for the intensity stream.
      \item split on-spill and off-spill to separate file streams
      \item split off-spill into triggered and pedestal.
    \end{enumerate}
  \item Where in the workflow will we merge each Trk+Cal file with it's matching CRV file?
    It could be in either RDM or Pass1; so we will need to discuss with the Pass1 team.
  \item Check with TDAQ: will you close an output file every subrun and open a new one? Will it be the same
    on all file streams or can it differ?  What's the plan to deal with a subrun transition when one
    event is very late arriving?  How does that last case affect RDM?
  \item At one time it looked like the raw data format for the CRV might be very inefficient
    because every header had to be present even if there was no associated data.
    One mitigation that was discussed was to reformat the raw data after it was in dCache and before it was written to tape.
    The other mitigation was to change the off-spill event duration to 100~$\mu$s; did that provide enough mitigation for off-spill events?
    For on-spill events the CRV data is small part of the whole; is it worthwhile to improve efficiency on a small part of the whole?
    What do we need to do here?  Does this belong in the risk registry?
  \item This document has some implicit assumptions.  Are they still correct?  Where should we write them down to make them explicit:
    \begin{enumerate}
      \item The unit of data processing will be an integer number of files; because of atomic subruns, this implies an integer number
        of complete subruns.
      \item Files must be small enough to be processed in under a few hours.
      \item The unit of checkpointing will be one grid job.
    \end{enumerate}
\end{enumerate}
