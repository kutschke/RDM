\chapter{Risk Registry}
\label{ch:RiskRegistry}

\section{Non-atomic Subruns from TDAQ}
\label{sec:Risk:NonAtomicSubruns}

The TDAQ group plans that each file stream created by TDAQ will have atomic subruns;
see Appendix~\ref{ch:AtomicSubrunsExtendedDefinition} for a definition of atomic subruns.
There is a risk that resource limitations will not allow this and, within some file streams,
events from one subrun will be spread over 2 or more files.  It will not be possible
to test the TDAQ design at scale until all equipment has been purchased and deployed.
At that time it will be possible to do at-scale tests of much of the system using simulated events
but there will remain a risk that the simulated events are not a good enough representation of data.
Taking taking cosmic ray data will not stress the system enough to discover unforeseen resource limitations,
This risk cannot be retired until we are taking beam data at design intensity.

Should this risk materialize Mu2e DPC will need to modify the bookkeeping
system to allow non-atomic subruns early in the workflow and modify the workflow
to restore atomic subruns as soon as practical.
There are no resources in the online system to merge files to restore atomic subruns
so that work will need to be done after the data is in the computer center.
The merging could be done either as part of RDM or as part of Pass1.

An alternative is to tighten the trigger algorithms to accept fewer events;
this risks loosing events that we really do want to keep and is the last alternative.
It might be an expedient temporary measure during commissioning.
