\chapter{Risk Registry}
\label{ch:RiskRegistry}

\section{CRV Data Split from the Trk+Cal Data}
\label{sec:Risk:NonAtomicSubruns}

The TDAQ group is concerned that bandwidth limitations and lack of buffering
space may require a different operating mode than described above.  If that
happens they have proposed this workaround: write the Trk+Cal data products
from each event to one file in the DLLS but write the CRV data products for
the same event to a separate file in the DLLS.  If this happens one
of the responsibilities of the Offline team will be to merge this these files.
The solution may be within the RDM system, in Pass1 or distributed across the two.
This section explains the problem and explains our understanding of the solution
proposed by the TDAQ team.

\section{Non-atomic Subruns from TDAQ}
\label{sec:Risk:NonAtomicSubruns}

The TDAQ group plans that each file stream created by TDAQ will have atomic subruns;
see Appendix~\ref{ch:AtomicSubrunsExtendedDefinition} for a definition of atomic subruns.
There is a risk that resource limitations will not allow this and, within some file streams,
events from one subrun will be spread over 2 or more files.  It will not be possible
to test the TDAQ design at scale until all equipment has been purchased and deployed.
At that time it will be possible to do at-scale tests of much of the system using simulated events
but there will remain a risk that the simulated events are not a good enough representation of data.
Taking taking cosmic ray data will not stress the system enough to discover unforeseen resource limitations,
This risk cannot be retired until we are taking beam data at design intensity.

Should this risk materialize Mu2e DPC will need to modify the bookkeeping
system to allow non-atomic subruns early in the workflow and modify the workflow
to restore atomic subruns as soon as practical.
There are no resources in the online system to merge files to restore atomic subruns
so that work will need to be done after the data is in the computer center.
The merging could be done either as part of RDM or as part of Pass1.

An alternative is to tighten the trigger algorithms to accept fewer events;
this risks loosing events that we really do want to keep and is the last alternative.
It might be an expedient temporary measure during commissioning.

\section{Major Network Failure}
\label{sec:majorNetworkFailure}

The CCD network group owns and maintains the network that runs from the Mu2e Hall to the computer center.
They have an annual budget for repairs to the network.
If there were serious damage to the network,
and if repairing it exceeded the available budget,
and if Mu2e needed a quick repair to maintain data taking,
then they have said that they would ask Mu2e to provide the money to pay for the repair.
This information came from Rafael Rocha.

\fixme{This needs to be in the MOU with CS.}

