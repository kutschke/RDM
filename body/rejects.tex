%
% This file holds fragments of text that I wrote but later decided that they did not belong
% in this document.  I am keeping them here to skim them later to see if I missed anything.
%

Each DAQ server has either one or two custom FGPAs mounted its PCIe bus;
these are called Data Transfer Controllers (DTC).
Each DTC is directly connected by optical fibers to a subset of the channels
in one detector subsystem;
it recieves data
sees a subset of the readout electronics from one detectors;

the DTCs is connected by optical fiber.


Data flows from the Tracker and Calorimeter into 36 of 40 DAQ servers;
each server sees a subset of the channels from one of the detectors.
More specifically, the data is received by Data Transfer Controllers (DTC) chips,
which are custom FPGAs that sit on the PCIe bus of each server,
either one or two DTCs per server.
The DTCs talk to each other via a dedicated high speed event building network switch
that is separate from the other networks that are part of the TDAQ world.
(each DTC has many optical fibers connected to it).
The cartoon picture is that each DTC recieves a stream of event fragments;
all of the DTCs exchange fragments
and the end result is that each DTC holds a buffer of complete events;
every event is present in exactly one DTC.
When a DTC has one full buffer of events it tells its host computer that the buffer is
available and it starts to fill the next buffer.

The host computer recieves the event buffer from the DTC
and makes it available as a source of events for trigger filter processes.

\fixme{ Instead of ``recieves'' do I want ``copies'', DMAs ...?}

The software that distrutes events the filter processes will usually send events
to filter processes running on the same DAQ server but it is smart enough to
load balance across all servers.

The trigger filter software looks at both the tracker and calorimeter information
and selects events of interest using many different criteria.
Each criterion is expressed as a series of \art modules run sequentially in
an \art ``trigger path''.  This is also referred to as a ``trigger line''.
For example one trigger line will select events that contain a track that is
consistent with being a downstream moving electron with a momentum in broad
windown near the signal window.  Other trigger lines will look for events
that useful for calibration.  Other trigger lines will prescale events
that fail one of trigger filter modules.  A snapshot of the currently
planned trigger lines can be found in Section~??\fixme{add reference}.

Events that pass the trigger, included events that from the

Two special cases: intensity stream and 



\fixme{reference is p20 of Mu2e-doc-38735, Ryan's talk at the June 2021 Collab meeting}


A large subset of the nodes in the DAQ farm have custom programmed
FPGAs connected to their PCIE buses;
these are called Data Transfer Controllers (DTCs).

------------------------------------------------------------------------------

\fixme{where does the next para belong?}

\art has a feature that allows users to define the maximum
number of events in one file or a maximum file size.
When this feature is enabled,
and when \art detects that a limit has been reached,
it closes the output file and opens a new one.
\art provides several options to ensure distinct names for these files.
Some experiments use these features to split the events from one
subrun over many files,
keeping each file small enough for convenient data handling and data processing.
Mu2e has decided not to do this.
Mu2e will choose the duration of a subrun to ensure that this is not needed.
The language is that Mu2e will produce ``atomic subruns''; there are some
subtlties to this that are discussed in Appendix~\ref{sec:AtomicSubruns}.


-------------------------------------------------------------------------
\section{Atomic Subruns}
\label{sec:AtomicSubruns}

A subrun is said to be ``atomic'' if all of the events in that subrun are found in a single \art file.
If subruns are atomic, some bookkeeping operations are simplified and Mu2e plans to exploit this;
therefore the TDAQ and RDM should aim to create atomic subruns as early as possible in the workflow.

Why ``as early as possible'' and not ``must''?

The base design of the TDAQ is to have one {\code artdaq::DataLogger} process that will
see all events and write them to disk; it may write several output files.
However it is possible that a single {\code artdaq::DataLogger} will not have the
resources to handle the load.  Should that happen the solution is to have two
{\code artdaq::DataLogger} processes, each of which sees about half of the events.
In this case all of the output files will be duplicated,
one for each {\code artdaq::DataLogger}.
In this case subruns are not atomic;
the RDM and DPC will need to work together to decide where in the workflow these two files
are joined to create atomic subruns.

There is an additional subtlety.
Section~\ref{sec:TagsIDsRunsSubRuns} described that Mu2e subruns will cover many MI cycles
and, therefore, will include both on-spill and off-spill events.
One option for the TDAQ, the option that is preferred by DPC,
is that TDAQ will write on-spill and off-spill events to separate files.
The Mu2e DPC anticipates only workflows in which these two files are treated separately;
another way of saying it is that there will be no jobs that read events from both of these files.
Therefore we can say that within each of these two file streams the design is for subruns to be atomic.
This is the usual meaning of ``atomic subrun'' throughout this document.

\subsection{Sparse Skims}

This section assumes that the reader is familiar with the \art concepts
of data products that are held by the {\code art::Run} object
and data products that are held by the {\code art::SubRun} object.
It also assumes that the reader is familiar with the \art concepts
of {\code art::Run} and {\code art::SubRun} fragments
and with the concept of aggregation for these fragments.
For information about these concepts see~\cite{RunAndSubRunProducts}
and~\cite{ProductAggregation}.


By default \art assumes that subruns are not atomic
but you can tell \art that subruns will be atomic by adding this fcl
parameter to the parameter set of the RootInput module

\cmd{ compactEventRanges : true}

When \art sees a beginRun transition, it reads all run data products from the input file.
When \art is configured for atomic subruns, it drops these data products
from memory at the matching endRun transition.
Similarly, when \art has beginSubRun transition, it reads all subrun data products from the input file.
When \art is configured for atomic subruns, it drops these data products
from memory at the matching endSubRun transition.
When \art is not configured for atomic subruns it does not drop these data products from
memory at the matching endSubRun transition; the reason is that 

On the other hand, when art is configured for non-atomic subruns

When \art is configured for non atomic subruns there is an extra memory cost;
for sparse skims this can dominate the memory usage of the process.
While this added memory cost is small early in the production, chain when
only a few subruns are seen by any given process, it's important to be
aware of this constraint when shaping the data for downstream processing.

To understand how the memory cost arises,
it is necessary to understand the structure of an art file.
Suppose that you have an art file with run:subrun ( 1:1, 1:2, 2:1, 2:2).
When you read the art file you will see the sequence of records shown
in table~\ref{tab:atomicsubruns}.

\begin{table}
\begin{center}
\caption[Example For Discussion of Atomic Subruns]{Example for the Discussion of Atomic Subruns}
\label{tab:atomicsubruns}
\begin{tabular}{l}\hline
  begin run record for run 1 \\
  begin subrun record for subrun 1:1 \\
  records for the events of subrun 1:1 \\
  end subrun record for subrun 1:1 \\
   begin subrun record for subrun 1:2 \\
   records for the events of subrun 1:2 \\
   end subrun record for subrun 1:2 \\
   end run record for run 1. \\ \hline
   begin run record for run 2. \\
   begin subrun record for subrun 2:1 \\
   records for the events of subrun 2:1 \\
   end subrun record for subrun 2:1 \\
   begin subrun record for subrun 2:2 \\
   records for the events of subrun 2:2 \\
   end subrun record for subrun 2:2 \\
   end run record for run 2. \\
   \hline
  \end{tabular}
\end{center}
\end{table}

Now suppose that your job reads the run and subrun scope data products at begin run/subrun time.
If \art is configured for atomic subruns, after processing the module and service
{\code endSubRun} calls, it will delete the subrun data products from memory. It can safely do this
because it knows that it will never see another event from that subrun.
Similar for the run scope data products.

On the other hand, if \art has been configured for non-atomic subruns, it will retain all subrun
and run data products in memory until the end of the job.
For a job that sees a small number of subruns this is usually not an important memory cost; but
for sparse skims it can be dominant.


There is a second consideration about atomic subruns:
for sparse skims there is significant memory savings if subruns are atomic.
While the RDM will not see sparse skims, the people developing it should know the big picture.


