\chapter{Introduction}

\label{ch:intro}

The Mu2e Raw Data Mover (RDM) System is an element of Mu2e Data Processing
and Computing (DPC)~\cite{DPC}, an L2 project within Mu2e Experiment Operations Plan~\cite{PEOP}.
Its purpose is to move data
that is produced by the online system to long term storage;
for most data, the long term storage will be files on tape
but, for some data, it will be in one of the offline databases.
Some data may also reside transiently in disk files so that
it is readily avaialble for the follow-on data processing steps.

Other functions of the RDM include:
\begin{enumerate}
\item Updating the file catalog to include meta-data and file location(s).
\item Any splitting/joining or other reshaping of files that is needed to match the needs of downstream processing.
\item Managing the free space in the online disk buffer
\item Copy/mirror subsets of the online databases to the offline databases
\item Move miscellaneous other data, such as the output of the Data Quality Monitoring (DQM) system, to long term storage.
\end{enumerate}

The current plan is that the offline data processing workflows will be driven by updates to the file catalog;
so the RDM does not need other hooks into those workflows.

%The operation and maintenance of the Mu2e building router and the network between the Mu2e Hall and the computer center
%is the responsibility of the Fermilab Core Computing Division (CCD).
%The DPS has the responsibility to be the interface between Mu2e and CCD regarding this network.
%Details of this responsbility are in Section~\fixme{reference the appropriate section}.


The main body of this document will describe a view of the
Mu2e Trigger and Data Acquisition system (TDAQ) as seen from the RDM perspective.
The cartoon picture is that TDAQ writes files to a disk buffer and RDM drains the disk buffer.
However, there are about 20 logical data streams, some tightly coupled to each
other, some loosely coupled to the others and others independent of the others.
Understanding the relationships among the data streams and their implications
for downstream processing are the starting point for designing the RDM.

This document includes additional information that is not written down concisely
in other places and is needed to frame the description of the data streams.


The RDM owns no hardware.  It uses hardware that is owned and maintained
by the Mu2e TDAQ group,
by the Fermilab Scientific Computing Division (SCD)
and by the Fermilab Core Computing Division (CCD).
The only M\&S items within RDM are to pay CCD for router maintenance,
as described in Appendix~\ref{app:RouterAndNetwork}.

In early planning for the Mu2e DPC, what is now called the RDM was called the Data Logging System.
However that name is too close to two related concepts in the online world:
the {\code artdaq::DataLogger} processes that run on the Data Logger nodes at the end of the TDAQ
processing chain.  The name was changed to RDM to avoid confusion with these other uses of Data Logger.

\section{Conventions Used in this Document}

In this document the word ``computer center'' is used as a collective noun for the
Feynman Computing Center (FCC) and the Grid Computing Center (GCC).
When it is important to distinguish the two, one of the two will be named explicitly.
This choice also allows future changes to the computing center infrastructure over the
lifetime of Mu2e.

The DPC is not part of the Mu2e Construction Project but it is part of Mu2e Operations
and pre-Operations.
In various places, this document refers to ``Mue groups'', such as the TDAQ group.
Sometimes this will mean an L2 group within the Mu2e Construction Project and at other
times it will mean a group within the Mu2e Operations or Pre-Operations organization.
Most of the time it will not be important to distinguish between these different meanings
of group; when it is important, it will be stated explicitly.


\chapter{Background Information}
\label{chap:BackgroundInfo}
This Chapter describes the view of the TDAQ and the computer center as seen by the RDM.

\section{Block Diagram}
\label{sec:BlockDiagram}

Figure~\ref{fig:blockdiagram} shows a block diagram of the major elements involved
in the data flow from the experiment hardware to long term storage.
All elements in the left hand dot-dashed box are located in the Mu2e Hall
and, except for the Mu2e building router, are the responsbility of the various Mu2e groups.
All elements in the right hand dot-dashed box are located in the computer center
and are the responsibility of SCD; the internal details of the SCD-managed resources
are not shown and the RDM will treat these resources as a interconnected, coherent whole.
The Mu2e building router and the optical fibre that connects that router
to the computer center are the responsibility of CCD;
see Appendix~\ref{app:RouterAndNetwork} for details.

\begin{figure}[tbp]
\centering
\includegraphics[width=0.9\textwidth]{figures/interface_with_TDAQ.pdf}
\caption[Block diagram of interfaces seen by the Mu2e RDM]{
  Block diagram of the interfaces seen by the Mu2e RDM.
  Not shown are \fixme{not shown}.  See the text for a discussion of these elements.}
\label{fig:blockdiagram}
\end{figure}

Data flows from the detectors, through the DAQ system into the TDAQ computer farm.
The five approved detectors are the Tracker (Trk), Calorimeter (Cal), the Cosmic Ray Veto system (CRV),
the Muon Stopping Target Monitor (MSTM) and the downstream Extintion Monitor (ExtMon).
A sixth detector has been proposed, the Production Target Scanning Monitor (PTSM),
which is drawn with a dotted box;
moreover it's primary use is to send rapid feedback to the accelerator control room
and it's not clear what, if any, data it will send through the TDAQ system.
In any event the data volume from that PTSM will be small enough that it will not be
a driver for the RDM.

A cartoon picture of the TDAQ computer farm is that it has 40 DAQ Server nodes
that talk to the detectors, build events and run trigger algorithms on those events.
The design is a deadtimeless streaming DAQ system that has no hardware trigger;
it reads every event from the detector hardware and sends all events to software filters
that run on the DAQ server nodes;
the software makes the trigger decision.
The present design is 20 software filter processes on each of the 40 DAQ server nodes, 800 total.
Events that pass the trigger are forwarded to a Data Logger node that will write the events
to files in the ``Data Logger Local Storage'', a RAID 0 disk array on the bus of the Data Logger node.
Appendix~\ref{app:DataLoggerLocalStorage} has the specs for and a discussion about this disk array.
In earlier DPC related talks and documents this was refered to as the ``48 hour disk buffer''.
In addition to the triggered events, a variety of prescaled, untriggered events will pass the
trigger and be handled the same as events that passed the trigger.
Finally, there will be several special data streams.

The base design is to have a single data logger node that receives data from all DAQ server nodes.
Should there be limitations within the TDAQ system, such as bandwidth or latency limits,
it may be necessary to have two or more Data Logger nodes, each seeing a fraction of each data stream.
How this affects RDM will be discussed in section \fixme{add reference}.

The Data Logger nodes will also have a process called a ``Dispatcher''
that requests events from the DataLogger process
and forwards them to clients.
The Dispatcher is designed to exert no back pressure on the DataLogger process
and, therefore, will normally see only a subset of the events.
Two of the clients foreseen for the Dispatcher are an Event Display and
a Data Quality Monitoring (DQM) system,
which will produce histograms, timelines etc that can be viewed in real time.
Periodically the DQM system will write it's histograms, log files etc to
files in the data logger local storage.  One of the jobs of the RDM will be
to move these files to long term storage.

All of the computing resources in the TDAQ farm are on a private subnet
and cannot been seen from the lab network.  Access in and out
of the private subnet will be via a gateway node that is connected to
both the private subnet and the lab network.  The gateway node is the responsibility
of the TDAQ group.

When the Mu2e Hall was built and provisioned, CCD personnel installed a dual router
and connected it to the lab optical fiber network.
RDM will move data from the Mu2e Hall
to the computer center using this router and the lab optical fibre network.
The router is standard item that CCD uses in many places around the lab; it stocks spares.
CCD is responsible for the maintenance of both the router and the network.
There are two M\&S items associated with the router: Mu2e pays a yearly
maintenance fee on the router and Mu2e is responsible to pay for replacement
of the router when the warrantee period expires.  Both are paid to CCD,
which has a lab-wide arrangement with the router vendor.
The specs of this router and additional details of Mu2e's arrangement with CCD
can be found in Appendix~\ref{app:RouterAndNetwork}.

For the foreseeable future Mu2e expects that the disk and tape services provided
by SCD will be dCache and ENSTORE.
At this writing Mu2e is using SAM as the file catalog system for files produced
by simulations, tests stands and Vertical Slice Tests (VST).
SCD plans to phase out SAM and replace it with a more modern system, RUCIO \fixme{reference};
RUCIO is already in use by CMS and by DUNE for Proto-DUNE single phase data.
It is likely that SAM will be phased out during the lifetime of Mu2e.
One of the questions to ask in the design of the RDM is
when to make the transition to RUCIO.

\section{On-Spill, Off-Spill and Near-Spill}

This section defines the concepts of on-spill, off-spill and near-spill running.

It is expected that, for most of it's running time,
Mu2e will share parts of the Fermilab accelerator complex with a neutrino experiment,
first, with NOvA and, later, with DUNE.
To keep this document easier to read it will discuss sharing with NOvA but sharing with DUNE is implied.
The change from NOvA to DUNE will not materially change the requirements of the RDM.

Sharing with NOvA drives the repeating cycle of Mu2e operations, a Main Injector (MI) cycle of
21 Booster (BO) cycles;
the duration of one Booster cycle is 1/15~s,
making the duration of one MI cycle 1.4~s.
Figure~\ref{fig:beamMacroStructure} shows a cartoon of the MI cycle.
From the Mu2e point of view, the MI cycle starts with a series of 8 spills;
within each spill, proton pulses arrive at Mu2e every 1695~ns;
the duration of a spill is about 43.1~ms, or about 25,500 pulses.
There is a gap of about 5~ms between spills.
The eigth spill is followed by a period of about 1020~ms,
during which no proton pulses arrive at Mu2e;
during this time, beam is delivered to NOvA
and beam is prepared for the next 8 spills to Mu2e.
Then the cycle repeats.

\begin{figure}[tbp]
\centering
\includegraphics[width=0.9\textwidth]{figures/ProtonBeamLongitudinalStructure2019-01-10_page6.pdf}
\caption[MacroStructure of the Proton Beam at Mu2e]{
  MacroStructure of the proton beam at Mu2e, taken from page~6 of
  the file ``Proton Beam Longitudinal Structure (.pdf)'' from
  Reference~\protect{\cite{beamTiming}}.  The figure is described in the text.}
\label{fig:beamMacroStructure}
\end{figure}

During the spills, Mu2e is running in the on-spill configuration.
During the 1020~ms no-beam period, Mu2e is running in the off-spill configuration.
These two configurations are discussed below.
It has not yet been decided if the 5~ms periods between spills will
be in the on-spill or off-spill configuration
but that does not materially affect the design of the RDM.
Table~\ref{tab:timescales} summarizes the time scales within one MI cycle.
\begin{table}
\begin{center}
\caption[Numerology of the Mu2e MI Cycle]{Numerology of the Mu2e MI Cycle.}
\label{tab:timescales}
\begin{tabular}{ll}\hline
   1/15~s & Period of one Booster cycle \\
   21     & Booster cycles within the MI cycle \\
   1.4~s  & Period of one MI cycle \\
   43.1~ms & Duration of one spill \\
   8       & Number of spills per MI cycle \\
    5~ms   & Duration of the period between two spills \\
   1020~ms & Duration of the off-spill period within one MI cycle \\
   1695~ns & Period of the Delivery Ring and the duration of one on-spill event\\
   100~$\mu$s & Duration of one off-spill event \\
   24.6\%     & Duty factor (total spill time divided by MI cycle duration)\\
   \hline
  \end{tabular}
\end{center}
\end{table}


In the on-spill configuration, one event will have  a duration of 1695~ns
and it records the information associated with the arrival of one proton pulse.
The trigger will be configured to select interesting events associated with the proton pulse
and to pre-scale non-triggering events in order to study the performance of the trigger.
In the off-spill configuration, one event will have a duration of 100~$\mu$s.
During the off-spill period, Mu2e will trigger on cosmic rays that
pass through the tracker and/or the calorimeter; it will also collect
pedestal data from many of the subsystems; some calibration procedures
may also take place during the off-spill period.

Mu2e expects that the accelerator super-cycle will consist of a
sequence of many of the above MI cycles,
with an occaisional other cycle mixed in.
For example there might be beam to MTEST about once per minute.
When this occurs, Mu2e will have an extended period of off-spill running.

There will also be periods in which the accelerator complex is not delivering
beam to Mu2e for an extended time, minutes, hours, days or weeks.
In some of these periods Mu2e will shutdown to perform maintenance or repairs;
in others Mu2e will execute calibration runs;
and in others Mu2e will collect data in off-spill mode for an extended period of time.

The Mu2e TDAQ team reserved space in their configuration packet to define
a third operational mode, that they have named near-spill.
This is to allow for the possibility that some subsystems may want to
configure themselves differently during the transition from on-spill to off-spill.
At this time there are no definite plans to use this mode but it is available if needed.

It's possible that the MI cycle might be changed to 22 BO cycles, which
will extend the duration of the off-spill period.
Also, Mu2e plans to start operations with a slightly different configuration:
4 spills instead of 8, each of a longer duration.
Neither of these materially changes the requirements for the RDM and
will not be discussed further.

There will be times when Mu2e is taking data but NOvA is not.
In such times, the base plan is that Mu2e will continue to take data using the same MI cycle.
There are several technical reasons that prevent Mu2e from reducing the off-spill period
in order to increase the duty factor.\footnote{
There are radiation safety limits on the average beam power;
the production target will have a reduced lifetime at higher average beam power;
the TDAQ system cannot accomodate a signficiantly higher event rate.
}

\section{Event WindowTags, EventIDs, Runs and Subruns}

Within the TDAQ system, events are identified by a 48 bit counter refered
to as the Event Window Tag (EWT);
this counter will be monotonically increasing and unique over the full Mu2e Run.
The EWT is distributed by the Clock Fan Out (CFO) system to all of the detector
readout electronics; it is the basis for event building.

Mu2e has chosen the \art event processing framework to host the algorithms
that will be run on Mu2e data.  The software filter processes that run
in the TDAQ farm will each be a separate \art process.  \art will also
be used for offline data processing.
In \art, events are uniquely identified by a 3 part identifier called an
{\code art::EventID}; the parts are named run number, subrun number
and event number.

The TDAQ system will translate each EWT to an {\code art::EventID}
before events are presented to the \art based software filter processes.
The EWT will be recorded as part of the data payload for each event.

The meaning of run and subrun are defined by Mu2e.
All that \art requires is that a subrun contain zero or more
events and that a run contain zero or more subruns.

The current plan is that runs will have a duration of a few hours
and that subruns will contain an integer number of MI cycles and
have a duration somewhere between 14 seconds and a few minutes.
This means that each subrun will contain both on-spill and
off-spill events.   The transition between subruns will
happen during the off-spill period so that each on-spill
period is contained within one subrun.

These durations are informed by the following.
Mu2e has designed a conditions management system
in which intervals of validity are a range of subruns;
such a range may, but need not, span runs.
Therefore subruns must be short enough to follow the most
rapidly changing conditions.
Mu2e TDAQ has been designed so that subrun transitions will be
deadtimeless but that data taking will briefly stop at run
transitions.  The TDAQ system will do some house keeping
on run boundaries, including resetting some hardware, reloading
some firmware and restarting some processes.

\fixme{Is this a good place to talk about atomic subruns and
  how that informs the choice of subrun duration. }


% Example
% Keep sections under development
\IfStrEq{\ISDRAFT}{YES}{

\chapter{A Chapter under development}
\label{ch:under_development}

This is a chapter still under development.
It is kept here to illustrate the use of the  ISDRAFT macro.

} % end `ISDRAFT = YES'

\chapter{Requirements}

This chapter lists requirements for the RDM system.
The requirements are dervived from the above firehose of information.

\begin{enumerate}
  \item Move data in a timely fashion
  \item prioritize recovery from down time.
\end{enumerate}

\chapter{Questions}

A list of questions that should be addressed in the design of the RDM:

\begin{enumerate}
\item Start by sticking with SAM and move to RUCIO later?  Or move to RUCIO now?
  How does this decision interact with requirements for ongoing simulation, VST
  and test stand work?
\end{enumerate}


\appendix

\chapter{Network Between Mu2e Building Router and Computer Center}
\label{app:RouterAndNetwork}.

The Mu2e building router is owned by CCD.
\begin{itemize}
\item specs; dual; auto fail over; channels; free to increase \#channels
\item Router is owned and maintained by CCD; stock item so they can replace quickly.
\item We pay yearly maintenance: \fixme{Amount in FY21 and inflation estimate}
\item We pay for replacement; we pay CCD and they do the work. \fixme{when is next replacement due; replacement cycle}
\item \fixme{make sure this is in MOU} Coverage.
\item \fixme{not sure who paid for the existing one.}
\end{itemize}


The lab network.
\begin{itemize}
\item Install and maintained by CCD.
\item Normal maintenance and repair is budgeted for in their ops budget
\item If there were an emergency repair that exceeded their budget they would come to us. \fixme{make sure this is covered in MOU}
\end{itemize}

\chapter{Data Logger Local Storage}
\label{app:DataLoggerLocalStorage}

\fixme{ Specs etc on the Data Logger Local Storage go here}.

\clearpage
\phantomsection
\addcontentsline{toc}{chapter}{Bibliography}
\printbibliography


%\cleardoublepage
%\printindex
