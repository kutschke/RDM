\chapter{Things to Know About Tapes}
\label{app:AboutTapes}

Some of the following information comes from email conversations with
Rafael Rocha (CCD) in late 2019 and early 2020.

As of summer 2021 Mu2e data is written to LTO8 tape cartridges,
Linear Tape Open (LTO) 8th generation~\cite{LTO}.
Each LTO8 cartridges holds 12~TB and as of fall 2019 cost \$140. per cartridge.
Historically a generation N LTO drive can write media from generations N and N-1
and can read data from generation N-2.
Historically CCD has skipped media generations so the next
media generation likely to be used at Fermilab will be LTO10,
which has a planned capacity of up to 36~TB per cartridge~\cite{LTORoadmap};
as of summer 2021 there is no timetable for the rollout of LTO10.

Until guidance from CCD changes, Mu2e will plan that we will start
data taking using LTO8 media and switch to LTO10 during the run.

The guidance from CCD is that the sweet spot for efficient use of LTO media
is around 1000 files per cartridge.  For LTO8 media this is about 1.2~GB per file.
If file sizes are signficiantly smaller, then a larger fraction of the
tape is taken up by inter-file gaps.

It may happen that the sweet spot for data processing is to have files that
are smaller than 1.2~GB per file.  To improve tape use efficiency CCD
provides a system called Small File Aggregation (SFA).

\fixme{Reference for SFA}.

A cartoon picture of SFA is the following.  Consider, for example,
that Mu2e needs to write files that are 300~MB in order to match
downstream processing needs.  In that case the SFA system will collect
4 files into a tar file and write that tar file to tape.  The SFA
system does the bookkeeping to know which file is in which tar file.
When we ask for a file to be restored from tape, the system will
identify which tar file needs to be restored from tape and restore that
tar file to temporary disk space.  It will then unpack that tar file
and restore the requested file to its dCache disk location.

All of the gymnastics with tar files is hidden from end users, who
only see that the files are stored on tape and are restored on demand.

Experience has shown that SFA works well with up to 10's of files
per tar file.  It works poorly with thousands of files per tar file.

\fixme{To add:}
\begin{enumerate}
  \item speed specs for seek/rewind, read.
  \item Mount time spec including robot motion
  \item What are our current SFA settings.  How do you look them up.
  \item Prestaging does not update the LRU date.  Nor does touching.  You must read at least one byte.
\end{enumerate}
