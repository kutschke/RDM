\chapter{About Tapes}
\label{app:AboutTapes}

Some of the following information comes from email conversations with
Rafael Rocha (CCD) in late 2019 and early 2020.

As of summer 2021 Mu2e data is written to LTO8 tape cartridges,
Linear Tape Open (LTO) 8th generation~\cite{LTO}.
Each LTO8 cartridges holds 12~TB and as of fall 2019 cost \$140. per cartridge.
Historically a generation N LTO drive can read/write media from generations N and N-1
and can read data from generation N-2.
Historically CCD has skipped media generations so the next
media generation likely to be used at Fermilab will be LTO10,
which has a planned capacity of up to 36~TB per cartridge~\cite{LTORoadmap};
as of summer 2021 there is no timetable for the transition to LTO10.

Until guidance from CCD changes, Mu2e will plan that we will start
data taking using LTO8 media and switch to LTO10 during the run.

The guidance from CCD is that the sweet spot for efficient use of LTO media
is around 1000 files per cartridge.  For LTO8 media this is about 1.2~GB per file.
If file sizes are significantly smaller,
then a larger fraction of the tape is taken up by inter-file gaps
and the capacity of one cartridge decreases.

It may happen that the sweet spot for downstream data processing is to
write files that are smaller than 1.2~GB per file.
To improve tape use efficiency CCD the recommended procedure is to use Small File Aggregation (SFA).

Experience from the past has shown that it's not wise to use SFA to manage
very large numbers of very small files.  Examples from past simulation campaigns
include the individual fcl files for each job, the job log files
and the small root TFileService files that were created by many jobs.
For every operation, there are per file overheads in addition to overheads
that scale with data volume.  Operations include: uploading the file
to tape backed dCache from it's previous location, registering the file
with SAM, updating SAM with tape locations.  While these overheads
are not part of SFA proper, they are needed to use SFA.
In addition SFA has a timeout after which it will write the current set of files to
tape, even if the size of the resulting tar file would still small.
Typically that timeout is 1 day.
This defeats the purpose of SFA since we are still writing small files to tape.

The preferred solution is to keep these files on disk until the end of the campaign,
make tar files ourselves
and upload those tar files to tape under the etc data tier.
This scheme only works for archival files, that is, for files that we expect to
read rarely, if at all.
If we want to keep redundant copies of these files until they are on tape,
we should do so by putting a second copy on dCache or perhaps on our NAS disks.

\section{File Families}
\label{sec:FileFamilies}

\section{Small File Aggregation}
\label{sec:sfa}

Most of the information in this section comes from an email conversation with
John Hendry (CCD) in September 2021.  See also the main SFA reference~\cite{SFA}.

Information about the configuration of active SFA for Mu2e file families can
be found at by searching for the string ``Mu2e'' on this page: \\
\hphantom{mm}
\href{https://www-stken.fnal.gov/cgi-bin/enstore\_sfa\_files\_in\_transition\_cgi.py}
     {https://www-stken.fnal.gov/cgi-bin/enstore\_sfa\_files\_in\_transition\_cgi.py}\\
You will need FNAL VPN access to access this page.
It is also linked from Mu2e computing operations page: \\
\hphantom{mm}
\href{https://mu2ewiki.fnal.gov/wiki/OfflineOps}{https://mu2ewiki.fnal.gov/wiki/OfflineOps}
at the link named ``Active SFA''.

The first column is the name of the SFA configuration and it contains a string that describes
the tape system on which the files will be written. We are currently writing to CD-LTO8G2.
This web page also includes some configurations with the name CD-LTO8F1; these were current
when we were writing to M8 media on LTO8 drives, at the time when LTO8 media was
not available.  Ignore these lines.  There are also configurations with the string
MIGRATION in their name; these are used by the jobs that are migrating files from older
generation media to LTO8 media. Ignore these too.

Each configuration contains 3 parameters that control the decision of when to make a tar file.
As files are accumulated, the creation of the tar file will be triggered by the earliest of
these events:
\begin{itemize}
  \item the number of waiting files reaches max\_files\_in\_pack
  \item the sum of the (uncompressed) size on disk of the waiting files reaches minimal\_file\_size
  \item the oldest of the waiting files is older than max\_waiting\_time seconds.
\end{itemize}
Once one of these limits is reached, the files will be queued for making a compressed tar
file and writing it to tape.  Any new files written after this will be accumulated towards
the next tar file.

If a file has a size of more than max\_member\_size it will not be included in a tar file;
instead it will be written to tape as an ordinary file.

The column resulting\_library identifies the disk space to be used for
the transient compressed tar file.
The field name wrapper describes the format of the volume and file labels written to the tape.

The web page is difficult to read.  For convenience, a snapshot of that information is given
in Table~\ref{tab:SFA}.  Be aware that web page is authoritative and the table may be out of date.

\begin{table}
\begin{center}
  \caption[SFA Parameters]{A Snaphot of the SFA Parameters on September 7, 2021.
    Consult the web page listed in the text for authoritative values. }
\label{tab:SFA}
\begin{tabular}{lrrrr}\hline
  \multicolumn{1}{c}{File Family} &
  \multicolumn{1}{c}{minimal\_file\_size} &
  \multicolumn{1}{c}{max\_files\_in\_pack} &
  \multicolumn{1}{c}{max\_waiting\_time} &
  \multicolumn{1}{c}{max\_member\_size}  \\
  &
    \multicolumn{1}{c}{(kB)} &
  &
    \multicolumn{1}{c}{(s)} &
    \multicolumn{1}{c}{(MB)}  \\  \hline
phy-etc & 4882812 & 3000 & 86400 & 300 \\
phy-nts & 4882812 & 3000 & 86400 & 300 \\
phy-sim & 7812500 & 3000 & 86400 & 300 \\
usr-etc & 4882812 & 3000 & 86400 & 300 \\
usr-nts & 9765625 & 2000 & 86400 & 400 \\
usr-sim & 4882812 & 3000 & 86400 & 300 \\
tst-cos & 4882812 & 3000 & 86400 & 300 \\
phy-ntd & 7812500 & 2000 & 86400 & 300 \\
phy-raw & 7812500 & 2000 & 86400 & 300 \\
phy-rec & 7812500 & 2000 & 86400 & 300 \\
usr-dat & 7812500 & 2000 & 86400 & 300 \\
  \hline
  \end{tabular}
\end{center}
\end{table}




\fixme{To add:}
\begin{enumerate}
  \item speed specs for seek/rewind, read.
  \item Mount time spec including robot motion
  \item Prestaging does not update the LRU date.  Nor does touching.  You must read at least one byte.
\end{enumerate}
