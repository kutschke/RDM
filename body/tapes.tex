\chapter{About Tapes}
\label{app:AboutTapes}

Some of the following information comes from email conversations with
Rafael Rocha (CCD) in late 2019 and early 2020.

As of summer 2021 Mu2e data is written to LTO8 tape cartridges,
Linear Tape Open (LTO) 8th generation~\cite{LTO}.
Each LTO8 cartridges holds 12~TB and as of fall 2019 cost \$140. per cartridge.
Historically a generation N LTO drive can read/write media from generations N and N-1
and can read data from generation N-2.
Historically CCD has skipped media generations so the next
media generation likely to be used at Fermilab will be LTO10,
which has a planned capacity of up to 36~TB per cartridge~\cite{LTORoadmap};
as of summer 2021 there is no timetable for the transition to LTO10.

Until guidance from CCD changes, Mu2e will plan that we will start
data taking using LTO8 media and switch to LTO10 during the run.

The guidance from CCD is that the sweet spot for efficient use of LTO media
is around 1000 files per cartridge.  For LTO8 media this is about 1.2~GB per file.
If file sizes are significantly smaller,
then a larger fraction of the tape is taken up by inter-file gaps
and the capacity of one cartridge decreases.

It may happen that the sweet spot for downstream data processing is to
write files that are smaller than 1.2~GB per file.
To improve tape use efficiency CCD the recommended procedure is to use Small File Aggregation (SFA).

Experience from the past has shown that it's not wise to use SFA to manage
very large numbers of very small files.  Examples from past simulation campaigns
include the individual fcl files for each job, the job log files
and the small root TFileService files that were created by many jobs.
For every operation, there are per file overheads in addition to overheads
that scale with data volume.  Operations include: uploading the file
to tape backed dCache from it's previous location, registering the file
with SAM, updating SAM with tape locations.  While these overheads
are not part of SFA proper, they are needed to use SFA.
In addition SFA has a timeout after which it will write the current set of files to
tape, even if the size of the resulting tar file would still small.
Typically that timeout is 1 day.
This defeats the purpose of SFA since we are still writing small files to tape.

The preferred solution is to keep these files on disk until the end of the campaign,
make tar files ourselves
and upload those tar files to tape under the etc data tier.
This scheme only works for archival files, that is, for files that we expect to
read rarely, if at all.
If we want to keep redundant copies of these files until they are on tape,
we should do so by putting a second copy on dCache or perhaps on our NAS disks.

\fixme{Reference for SFA}.

\fixme{To add:}
\begin{enumerate}
  \item What are our current SFA settings.  How do you look them up.
  \item speed specs for seek/rewind, read.
  \item Mount time spec including robot motion
  \item Prestaging does not update the LRU date.  Nor does touching.  You must read at least one byte.
\end{enumerate}
