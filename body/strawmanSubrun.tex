\chapter{Strawmen for Model Building}
\label{ch:strawmen}

In order to develop model for data volume and data handling it is necessary
to define the duration of one Subrun and the running schedule over the course
of a year.  This chapter specifies strawman models for both.

\section{Strawman Subrun}
\label{sec:strawmanSubrun}

For purposes of this document it is assumed that supercycle of the accelerator
complex will be 40 MI cycles followed by 4 seconds of beam to MTEST.  This
corresponds to 56~s of MI cycles plus 4 seconds of extended off-spill running,
for a total duration of 1~minute.  The subrun transition will occur sometime
near the end of the 4 seconds of off-spill running.

\section{Strawman Year}
\label{sec:strawmanYear}

For purposes of this document it is assumed that the operational plan for
the accelerator complex is:
\begin{enumerate}
  \item 42~weeks per year of accelerator operations; the remaining time will be the yearly accelerator shutdown.
    \begin{enumerate}
    \item During the shutdown Mu2e will take cosmic ray data half of the time and be off entirely half of the time.
    \end{enumerate}
  \item 20/21 shifts per week of accelerator operations; the other shift is for planned maintenance and for R\&D.
  \item Mu2e willl collect beam data with an efficiency of 90\% during accelerator operations;
    the other 10\% time will be spent resolving technical problems and in run transitions.
  \item This gives an overall yearly data taking efficiency of 69\% plus an additional 10\% of the time taking
    cosmic ray data during the shutdown.
\end{enumerate}

\fixme{In item 3, 90\% seems high.}

