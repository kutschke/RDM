\chapter{Introduction}
\label{ch:intro}

\section{Overview}
\label{sec:overview}

The Mu2e Raw Data Mover (RDM) System is an element of Mu2e Data Processing
and Computing (DPC)~\cite{DPC}, an L2 project within Mu2e Experiment Operations Plan~\cite{PEOP}.
Its purpose is to move data
that is produced by the online system to long term storage;
for most data, the long term storage will be files on tape in the computer center
but, for some data, it will be in one of the offline databases.
Some data may also reside transiently in disk files in the computer center so that
it is readily available for the follow-on data processing steps.

The full list of computing functions provided by the RDM is:
\begin{enumerate}
\item Move data from the online disk buffer to long term storage.
\item Update the file catalog to include meta-data and file location(s).
\item Manage the free space in the online disk buffer.
\item Manage the free space in the parts of dCache that are critical to data taking.
\item Copy/mirror subsets of the online databases to the offline databases.
\item Copy miscellaneous other files, such as the output of the online Data Quality Monitoring (DQM) system, to long term storage.
\item Provide tools to manage and monitor the RDM system.
\item Any splitting/joining or other reshaping of files that is needed to match the needs of downstream data processing.
\item If needed, inform the follow-on workflows when work is ready for them to do.
\end{enumerate}


%The operation and maintenance of the Mu2e building router and the network between the Mu2e Hall and the computer center
%is the responsibility of the Fermilab Core Computing Division (CCD).
%The DPC has the responsibility to be the interface between Mu2e and CCD regarding this network.
%Details of this responsbility are in Section~\fixme{reference the appropriate section}.

The remainder of Chapter~\ref{ch:intro} will identify the stakeholders in this work,
describe some conventions used in this document and
state some prerequisites that will be needed to understand it.
Chapter~\ref{ch:SelectedDetails}  of will describe a view of the
Mu2e Trigger and Data Acquisition system (TDAQ) as seen from the RDM perspective;
it will also describe some elements of the downstream data processing as seen
from the RDM perspective.
Chapter~\ref{ch:DataStreamsAndFileStreams} will define the ideas of data streams and file streams.
Chapter~\ref{ch:data_streams} will give a strawman list of data streams.
Chapter~\ref{ch:file_streams} will give a strawman list of file streams.
Chapter~\ref{ch:questions} lists some questions that will need to answered when designing the system.
Chapter~\ref{ch:requirements} states the requirements for the RDM system.
Chapter~\ref{ch:RiskRegistry} describes risks that are known at this time.
The document includes several appendices that provide additional details.

%The cartoon picture is that TDAQ writes files to a disk buffer and RDM drains the disk buffer.
%However, there are about 30 data streams, some tightly coupled to each
%other, some loosely coupled to the others and others independent of the others.
%The data streams will be grouped into a smaller number of files.


The RDM owns no hardware.  It uses hardware that is owned and maintained
by the Mu2e TDAQ group,
by the Fermilab Scientific Computing Division (SCD)
and by the Fermilab Core Computing Division (CCD).
The only M\&S items within RDM are to pay CCD for maintenance on the Mu2e building router,
and for end-of-service-life replacement of that router;
see Appendix~\ref{app:RouterAndNetwork} for details.

In early planning for the Mu2e DPC, what is now called the RDM was called the Data Logging System.
However that name is too close to two related concepts in the online world:
the {\code artdaq::DataLogger} processes that run on the Data Logger nodes.
To avoid confusion with these other uses of ``Data Logger'',
the name of this task was changed to RDM.

\section{Milestones}
\label{sec:Milestones}

For reference, the milestones related to the Mu2e Construction Project and Mu2e Operations will
be discussed in Section~\ref{ssec:ExternalMilestones}.
The only hard milestone for RDM that is set by these milestones is at the start of
Commissioning with Beam in early 2025.
DPC is developing internal milestones to ensure that all DPC deliverables are
complete and well tested prior to that deadline.
Section~\ref{ssec:ExternalMilestones} presents the internal milestones.

\subsection{External Milestones}
\label{ssec:ExternalMilestones}

Mu2e Project is being rebaselined, which means that the external milestones tied to the Mu2e
Project and Operations schedules are not well known.
Estimates for the most important dates are given in Table~\ref{tab:externalmilestones};
these dates have been extracted from
page 18 of the Ron Ray's presentation at the June 2021 Mu2e Collaboration Meeting~\cite{ProjectUpdate202106}
and page 35 of Greg Rakness's presentation at the IPR held in February 2021~\cite{OperationsIPR}.

\begin{table}
\begin{center}
\caption[Estimates for External Milestones]{Estimates for External Milestones; see the text for references.}
\label{tab:externalmilestones}
\begin{tabular}{ll}\hline
  Jan 1, 2023 & Calorimeter fully assembled in Mu2e Hall; begin cosmic ray system test. \\
  Jul 1, 2023 & Tracker fully assembled in hall; continue cosmic ray system test. \\
  Oct 1, 2023 & Detector KPP; cosmic ray system test complete. \\
  Jan 1, 2025 & Begin commissioning with beam. \\
  Apr 1, 2026 & Begin physics operations. \\
   \hline
  \end{tabular}
\end{center}
\end{table}


The cosmic ray system test will take place with the detector in the extracted position.
Early on, only the calorimeter will be included in the test.
During the last few months of the test,
it will include the tracker, the calorimeter and several modules of the CRV system
arranged above the other two detectors to form a cosmic ray trigger.
All will be read out using the TDAQ system.

\fixme{Do we expect any CRV modules to be part of the calorimeter-only portion of the cosmic ray system test?}


In order to satisfy the Detector Key Performance Parameters (KPP)~\cite{Mu2eKPPs},
both the threshold and the objective versions,
it is not necessary to have a functioning RDM.
It is sufficient for the data to reach the disk buffer in the online system
and to be reconstructed by code running on the online system.
Never-the-less the goal of the RDM team is to have at least a minimal RDM system
in place prior to the start of the cosmic ray system test
and to use that test as an opportunity to test the RDM system.

The 15 month period between the Detector KPP and the start of commissioning with beam will be used
to complete the solenoid installation,
map the magnetic field,
insert the detector train into the Detector Solenoid (DS),
install shielding
and install the CRV system.
During this period the detector will take cosmic rays whenever possible.
These will fall into two categories:
brief runs to test if any functionality has been impaired by recent activities;
long runs while other activities are taking place in the Mu2e Hall.
The long runs will be used to acquire a large cosmic ray sample to
commission, calibrate and align the detector.

In summary, the one external hard deadline is to have a fully functional RDM system before the start of
Commissioning with Beam.
The next section will discuss the internal milestones set to ensure that this external milestone is met.

\subsection{Internal Milestones}
\label{ssec:InternalMilestones}

Table~\ref{tab:internalmilestones} presents a strawman for internal milestones.
The pattern is to plan an Operational Readiness Review 3 months prior to each external
milestone and to set an internal milestone 3 months before that.
Part of job that is being defined by this document is to define the content of each phase.

\fixme{These are still strawman milestones.  We need to discuss them and define them more clearly.
  These are aggressive deadlines.  I am still anxious that we won't have the staff to meet aggressive deadlines; so why set them?}

\begin{table}
\begin{center}
\caption[Strawman Internal Milestones]{Strawman Internal Milestones}
\label{tab:internalmilestones}
\begin{tabular}{ll}\hline
  Jul 1, 2022 & Phase 1 system; ready for Calorimeter only portion of the cosmic ray system test. \\
  Oct 1, 2023 & ORR for Phase 1 system. \\
  Jan 1, 2023 & Phase 2 system; ready for for the full cosmic ray system test. \\
  Apr 1, 2023 & ORR for Phase 2 system. \\
  Jul 1, 2024 & Phase 3 system; ready for commissioning with beam. \\
  Sep 1, 2024 & ORR for Phase 3; \\
   \hline
  \end{tabular}
\end{center}
\end{table}


Notes to add:
\begin{itemize}
  \item What about Tracker VST and other VSTs?
  \item Use VSTs to test prototypes for phase 1
  \item Use Phase N-1 to test Phase N.
  \item While running with cosmic rays, expect low data volume but frequent changes to details.
    Can use this to demonstrate agility.
\end{itemize}


\section{Stakeholders}
\label{sec:stakeholders}

The stakeholders are:
\begin{enumerate}
  \item The Mu2e Operations organization.  The head is Greg Rakness.
  \item The DPC organization within Mu2e Ops.  The head is Rob Kutschke.
  \item The RDM team within DPC; not yet formed.  The role is being filled for now by Rob Kutschke and Ray Culbertson.
  \item The Pass1 team within DPC; not yet formed. The role is being filled for now by Rob Kutschke and Ray Culbertson.
  \item The TDAQ L2 within the Mu2e Construction Project.  The L2 manager is Ryan Rivera.  Eric Flumerfelt is also a critical member \fixme{Is Eric an L3?}.
  \item The TDAQ organization within Mu2e Ops. The head is Bertrand Echenard.
  \item The Mu2e Trigger group within the Mu2e Collaboration.  The co-conveners are Bertrand Echenard and Giani Pezzullo.
  \item The Mu2e Computing and Software group.  The co-conveners are Dave Brown (LBL) and Rob Kutschke.
  \item The Mu2e sub detector groups.  Are we correctly describing their raw data and what they want us to do with it?  \fixme{Ask leads to name delegates}
    \begin{enumerate}
      \item Tracker: Bob Tschirhart or delegate
      \item Calorimeter: Stefano Misceti or delegate
      \item CRV:  Craig Dukes or delegate
      \item ExtMon: Matthew Jones or delegate
      \item MSTM: Jim Miller or delegate
      \item PTSM: Kevin Lynch or delegate
    \end{enumerate}
  \item Representatives from ongoing DAQ work, such as the Tracker VST; the contact is Richie Bonventre.
  \item The Mu2e collaboration. The collaboration will be part of deciding the big picture needs for downstream data processing.
    Co-spokespersons Jim Miller and Bob Bernstein.
\end{enumerate}

\section{Conventions Used in this Document}
\label{sec:conventions}

In this document the word ``computer center'' is used as a collective noun for the
Feynman Computing Center (FCC) and the Grid Computing Center (GCC).
When it is important to distinguish the two, one of the two will be named explicitly.
This language also allows for changes to the computing center infrastructure over the
lifetime of Mu2e.

The DPC is not part of the Mu2e Construction Project but it is part of Mu2e Operations
and Pre-Operations organizations.
In various places, this document refers to ``Mu2e groups'', such as the TDAQ group.
Sometimes this will mean an L2 group within the Mu2e Construction Project and at other
times it will mean a group within the Mu2e Operations or Pre-Operations organizations;
and at other times it will mean a group within the collaboration organization.
Most of the time it will not be important to distinguish between these different meanings
of group; when it is important, it will be stated explicitly.

The phrase ``detector readout electronics'' is intended as collective noun for all elements
up to and including the Read Out Controller (ROC) chips; it does not include
the Data Transfer Controllers (DTC).

This document uses the terminology ``primary triggers'' and ``support triggers'';
these ideas are defined in reference~\cite{TriggerSU2020}.

\section{Units}
\label{sec:units}

\fixme{TB vs TiB, MB vs MiB, Gbs vs Gibs and so on}.

\section{Computing Costs}

\fixme{Reference Stu's document.  As of this writing we are
directly paying for any of this.}

\section{Prerequisites}
\label{sec:prerequisites}

This document assumes that the reader has basic familiarity with the \art event processing framework~\cite{ARTWORDPRESS}, including:

\begin{enumerate}
\item The ideas of events, subruns and runs.
\item {\code art::EventID}
\item Event, run and subrun  data products.
\item Run and subrun data product fragments; aggregation of run and subrun fragments.
\item The notion of an atomic subrun: see appendix~\ref{ch:AtomicSubrunsExtendedDefinition} for a definition and additional discussion.
\item Modules
\item Filter modules and paths
\end{enumerate}

It is a assumed that the reader has basic familiarity with SAM, dCache, ENSTORE~\cite{ENSTORE} and submitting grid jobs.
This includes understanding the ideas of File Families and Small File Aggregation (SFA)~\cite{SFA}.
Some additional details about tapes, including SFA configurations, are given in Appendix~\ref{app:AboutTapes}.

It is assumed that the reader knows the Mu2e standards and practices regarding SAM file names\cite{Mu2eSAM}
location of files in tape backed dCache\cite{FileNames}, and file families.
