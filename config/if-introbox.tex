% \include{lbnepaper/introbox.tex}
%
% provides an "intro" or "executive summary" box.

% Use in a content file as:
% \begin{introbox}
%   ...content...
% \end{introbox}

\usepackage[xcolor,framemethod=tikz,nobreak=false]{mdframed}%true]{mdframed}
\usetikzlibrary{shadows}


%\definecolor{introcolor}{rgb}{255,255,0}
\definecolor{introcolor}{cmyk}{0.16, 0.00, 0.51, 0.36} %sciopp-light-moss-green

\newcommand{\IntroBackgroundColor}{introcolor}
\newcommand{\IntroLineColor}{introcolor}

% An "intro" frame to use for procedures for which the steps should remain together.
% Place content in an "introbox" environment.
\mdfdefinestyle{introstyle}{frametitle=}
\mdfapptodefinestyle{introstyle}{linecolor=\IntroLineColor}
\mdfapptodefinestyle{introstyle}{backgroundcolor=\IntroBackgroundColor}
\mdfapptodefinestyle{introstyle}{roundcorner=7pt}
\mdfapptodefinestyle{introstyle}{outerlinewidth=1pt}
\mdfapptodefinestyle{introstyle}{frametitlefont=\normalfont\ttfamily}
\mdfapptodefinestyle{introstyle}{frametitlealignment=\centering}
\mdfapptodefinestyle{introstyle}{innerleftmargin=15pt}
\mdfapptodefinestyle{introstyle}{innerrightmargin=15pt}
\mdfapptodefinestyle{introstyle}{innertopmargin=15pt}
\mdfapptodefinestyle{introstyle}{innerbottommargin=15pt}

\newmdenv[style=introstyle]{introbox}



